\documentclass{article}
\usepackage[utf8]{inputenc}
\usepackage[english]{babel}

\usepackage[dvipsnames]{xcolor,soul}
\usepackage[framemethod=TikZ]{mdframed}

%%%%%%%%%%%%%%%%%%%%%%%%%%%%%%%%%%%%%%%%%%%%%%%%%%%
%%%%%           CUADROS DE TEXTO              %%%%%
%%%%%%%%%%%%%%%%%%%%%%%%%%%%%%%%%%%%%%%%%%%%%%%%%%%
\newcounter{theo}[section]\setcounter{theo}{1}
\renewcommand{\thetheo}{\arabic{theo}}
\newenvironment{theo}[2][]{%
\refstepcounter{theo}%
\ifstrempty{#1}%
{\mdfsetup{%
frametitle={%
\tikz[baseline=(current bounding box.east),outer sep=0pt]
\node[anchor=east,rectangle,fill=red!40]
{\strut Teorema~\thetheo};}}
}%
{\mdfsetup{%
frametitle={%
\tikz[baseline=(current bounding box.east),outer sep=0pt]
\node[anchor=east,rectangle,fill=red!30]
{\strut Teorema~\thetheo:~#1};}}%
}%
\mdfsetup{innertopmargin=10pt,linecolor=red!30,%
linewidth=2pt,topline=true,%
frametitleaboveskip=\dimexpr-\ht\strutbox\relax
}
\begin{mdframed}[]\relax%
\label{#2}}{\end{mdframed}}

\begin{document}

This example shows different examples on how to use the \texttt{xcolor} package 
to change the colour of elements in \LaTeX.

\begin{itemize}
\color{ForestGreen}
\item First item
\item Second item
\end{itemize}

\noindent
{\color{RubineRed} \rule{\linewidth}{0.5mm} }

The background colour of some text can also be \textcolor{red}{easily} set. For 
instance, you can change to orange the background of \colorbox{BurntOrange}{this 
text} and then continue typing.

\begin{theo}[Pythagoras' theorem]{thm:pythagoras}
In a right triangle, the square of the hypotenuse is equal to the sum of the squares of the catheti.
$$a^2+b^2=c^2$$
\end{theo}
In mathematics, the Pythagorean theorem, also known as Pythagoras' theorem (see theorem \ref{thm:pythagoras}), is a relation in Euclidean geometry among the three sides of a right triangle.


\sethlcolor{pink}  
The text given here is the \hl{highlighted text in pink.}\\  
  
\sethlcolor{orange}  
The text given here is the \hl{highlighted text in orange.}  

\end{document}