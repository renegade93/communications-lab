%%%%%%%%%%%%%%%%%%%%%%%%%%%%%%%%%%%%%%%%%%%%%%%%%%%%%%%%%%%%%%%%%%%%%%%%%
% Antes de correr el código:
% 1. Ingresar al Menu
% 2. Cambiar la opción "Compiler" a XeLaTeX
% 3. Cambiar la opción "TeX live version" a 2020 (opacidad de la imagen)
%%%%%%%%%%%%%%%%%%%%%%%%%%%%%%%%%%%%%%%%%%%%%%%%%%%%%%%%%%%%%%%%%%%%%%%%%
\documentclass[10pt]{article}
\usepackage[T1]{fontenc}
\usepackage[utf8]{inputenc}
\usepackage[english]{babel}
\usepackage{listings}
\lstset{language=R}
\usepackage[a4paper]{geometry}
\usepackage[dvipsnames]{xcolor}
\usepackage[framemethod=TikZ]{mdframed}
\usepackage{graphicx,tikz}
\usepackage{array}
\usepackage{float}
\usepackage{tocloft}
\setlength{\cftsecnumwidth}{2em}
\usepackage{xspace}
\usepackage{tcolorbox}
\tcbuselibrary{listings}

% code
\usepackage{listings}
\usepackage{xcolor}

\lstset{
  basicstyle=\ttfamily\small,
  backgroundcolor=\color{gray!10},
  frame=single,
  breaklines=true,
  columns=fullflexible,
  showstringspaces=false
}

\newtcblisting{cc}{
  listing only,
  nobeforeafter,
  after={\xspace},
  hbox,
  tcbox raise base,
  fontupper=\ttfamily,
  colback=lightgray,
  colframe=lightgray,
  size=fbox
  }

\geometry{top=2.54cm, bottom=2.54cm, left=2.54cm, right=2.54cm}

\usepackage{url}
\usepackage{lipsum} 
\usepackage{wrapfig}
\usepackage{subcaption}
\usepackage{multicol}

%==========================================
%======     FUENTE PARA CÓDIGOS      ======
%==========================================
\definecolor{codegreen}{rgb}{0,0.6,0}
\definecolor{codegray}{rgb}{0.1,0.1,0.1}
\definecolor{backcolour}{rgb}{0.98,0.98,0.98}

\lstdefinestyle{mystyle}{
  backgroundcolor=\color{backcolour},   
  commentstyle=\color{codegreen},
  keywordstyle=\color{blue},
  numberstyle=\tiny\color{codegray},
  stringstyle=\color{codegreen},
  basicstyle=\ttfamily\footnotesize,
  breakatwhitespace=false,         
  breaklines=true,                 
  captionpos=b,                    
  keepspaces=true,                 
  numbers=left,                    
  numbersep=5pt,                  
  showspaces=false,                
  showstringspaces=false,
  showtabs=false,                  
  tabsize=2
}

%==========================================
%==========     ESTILO TITLE     ==========
%==========================================
\newcommand{\City}[1]{\def\City{#1}}

\makeatletter         
\renewcommand\maketitle{
\begin{flushleft}
{\textcolor{black}{\huge \bfseries \@title }}\\[1ex]
\rule{\textwidth}{0.6pt}\\
\end{flushleft}
\vspace{-0.5cm}

\begin{flushleft}
\textcolor{black}{{\large  \@author} }\\[2ex]
\end{flushleft} } % Note the extra }
\makeatother

%==========================================
%==========    ESTILO CAPTION    ==========
%==========================================
\usepackage{caption}
\captionsetup[table]{name=Tabla ,textfont={it}, labelfont={bf},
                     justification=centering,
                     width =\dimexpr \textwidth-0.5cm\relax}
\captionsetup[figure]{textfont={it}, labelfont={bf},
                      justification=centering, skip=2pt,
                      belowskip=-5pt}
                      
%==========================================
%==========     ESTILO ITEM      ==========
%==========================================
\renewcommand{\labelitemi}{$\bullet$} 
\renewcommand{\labelitemii}{$\circ$} 
\renewcommand{\labelitemiii}{$\cdot$} 

%==========================================
%===    LINKS (Agregar Hyperlinks)     ====
%==========================================
\usepackage[style=apa,
            urldate=long]{biblatex} 
\addbibresource{Bib.bib}

\DeclareSourcemap{
  \maps[datatype=bibtex]{
    \map{
      \step[fieldsource=note, final]
      \step[fieldset=addendum, origfieldval, final]
      \step[fieldset=note, null]}}
}

\DefineBibliographyStrings{english}{urlseen = {Accessed }    
}

\usepackage[colorlinks=true,linkcolor=RoyalBlue,
            citecolor=RoyalBlue,urlcolor=RoyalBlue]{hyperref}

%==============================================================
%==============================================================
\title{ }

%%%%%%%%%%%%%%%%%%%%%%%%%%%%%%%%%%%%%%%%%%%%%%%%%%%%%%%%%%%%%%%
%%%%%%%%%%%%                 INICIO                %%%%%%%%%%%% 
%%%%%%%%%%%%%%%%%%%%%%%%%%%%%%%%%%%%%%%%%%%%%%%%%%%%%%%%%%%%%%%
\begin{document}

\begingroup
\let\clearpage\relax % prevent extra page breaks
\thispagestyle{empty}
\begin{center}
{\huge \bfseries Universidad de los Andes}

\vspace{25pt}
{\LARGE \bfseries Departamento de Ingeniería de Sistemas}

\vspace{15pt}
\includegraphics[width=100pt]{images/logo.png} 

\vspace{35pt}
{\LARGE \bfseries Laboratorio \#4: Protocolo de enrutamiento dinámicos con redes IPv4 e IPv6}
\vspace{55pt}

{\Large \bfseries ISIS3204 - Infraestructura de Comunicaciones}


\vspace{100pt}
{\Large \bfseries Grupo 3: }

\end{center}

\begin{flushleft}
  \setlength{\parskip}{0pt}
  \setlength{\itemsep}{0pt}
  \hspace*{4cm}\large\bfseries Juan Esteban Quiroga - 202013216

  \hspace*{4cm}\large\bfseries Juan Manuel Rodriguez - 202013372

  \hspace*{4cm}\large\bfseries Andres Felipe Ortiz - 201727662
\end{flushleft}

\begin{center}
\vspace{60pt}
\Large\bfseries 2025-10
\end{center}

\mbox{}
\endgroup

\clearpage

\tableofcontents
\clearpage

\renewcommand{\thesection}{\arabic{section}}
\section*{Introducción}
\addcontentsline{toc}{section}{Introducción}

El presente laboratorio tiene como objetivo principal \textbf{comprender y aplicar los protocolos de enrutamiento dinámico RIP y OSPF en entornos IPv4 e IPv6}, destacando sus diferencias estructurales y operativas. A través de la configuración práctica de routers Cisco en el simulador \textit{Cisco Packet Tracer}, se busca afianzar los conceptos teóricos sobre el funcionamiento de los protocolos de vector distancia y de estado de enlace, así como el proceso de intercambio de información de enrutamiento y la construcción de tablas de rutas en topologías multi-router.

Durante el desarrollo de la práctica se implementan diferentes escenarios de red que integran el direccionamiento IP, la configuración de interfaces, y la activación de protocolos de enrutamiento dinámico. Esto permite observar cómo los routers actualizan sus tablas de enrutamiento de manera automática y cómo se comporta la red ante variaciones en la topología o fallos en los enlaces.

El laboratorio permite al estudiante:
\begin{itemize}
    \item Diferenciar los principios de funcionamiento de \textbf{RIP (Routing Information Protocol)} y \textbf{OSPF (Open Shortest Path First)}, tanto en su versión IPv4 como IPv6.
    \item Configurar y verificar el intercambio de rutas dinámicas entre múltiples routers, observando métricas como el conteo de saltos y el costo por ancho de banda.
    \item Comprender el proceso de \textbf{asignación jerárquica de direcciones IP}, el uso de máscaras de subred y prefijos en IPv6, así como la importancia del direccionamiento correcto en la conectividad de red.
    \item Analizar el impacto de los protocolos de enrutamiento en la convergencia de la red y en la eficiencia del encaminamiento de paquetes.
\end{itemize}

De esta forma, la práctica integra los conceptos fundamentales del \textbf{nivel de red del modelo OSI}, brindando una experiencia completa que abarca el diseño, configuración y validación de una infraestructura de comunicación moderna y funcional.

%==============================================================
%=======================   1   ================================
%==============================================================
\section{Modelo Publicador–Suscriptor}

\begin{thebibliography}{9}
  \bibitem{kurose_ross}
  Computer Networking, a top-down approach. James Kurose, Keith Ross. Addison-Wesley, 6th ed.
\end{thebibliography}

\end{document}
